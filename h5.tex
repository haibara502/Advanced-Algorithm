\documentclass{article}

\usepackage{graphicx}
\usepackage{float}
\usepackage{amsthm}
\usepackage{algorithm}
\usepackage{algorithmicx}
\usepackage{algpseudocode}
\usepackage{amsmath}

\usepackage{amssymb}
\newenvironment{claim}[1]{\par\noindent\underline{Claim:}\space#1}{}
\newenvironment{guess}[1]{\par\noindent\underline{Guess:}\space#1}{}
\usepackage{amsthm}

\usepackage{geometry}
\geometry{left=2.5cm,right=2.5cm}

\title{Homework5}
\author{Qinyun Song}
\date{}

\begin{document}
	\maketitle

	\section{Markov's inequality}
		For a random variable $X$ and $X = -9$ with probability $0.5$ while $X = 11$ with probability $0.5$. Then we can see that \begin{equation}
		E[X] = 0.5 \times -9 + 0.5 \times 11 = 1
		\end{equation}
		And the probability \begin{equation}
			Pr \left[ X >  10 \right] = 0.5 = \frac{1}{2}
			\end{equation}

	\section{Comparing concentration inequalities}
		\begin{enumerate}
			\item The expectation of $X$ is \begin{equation}
				E\left[X\right] = 1 \times \frac{1}{3} + 0 \times \frac{2}{3} = \frac{1}{3}
				\end{equation}
				Then from the \emph{Markov's inequality}, we can see that \begin{equation}
				Pr \left[ X > \frac{t}{3}\right] \leq \frac{1}{t}
				\end{equation}
			\item We can calculate the variance of $X$ as \begin{equation}
					var(X) = N \times \frac{1}{3} \times \frac{2}{3} = \frac{2N}{9}
				\end{equation}
			\item Define variable $y_i = x_i - \frac{1}{3}$. Then we can see that $E\left[y_i\right] = 0$. We can see that \begin{equation}
			y_i = \frac{2}{3} wp \frac{1}{3} and y_i = -\frac{1}{3} wp \frac{2}{3}
			\end{equation}
			Thus we know that \begin{equation}
				var(Y) = E\left[Y^2\right] = \frac{4}{9} \times \frac{1}{3} + \frac{1}{9} \times \frac{2}{3} = \frac{6}{27} = \frac{2}{9}
			\end{equation}
			So for the std-dev, we have \begin{equation}
				std-dev(Y) = \sqrt{var(Y)} = \frac{\sqrt{2}}{3}
			\end{equation}
			If we want $X > \frac{2N}{3}$, then we need $Y > \frac{2N + 1}{3}$. Then by using the \emph{Chebychev's inequality}, we know that \begin{equation}
				Pr\left[ X > \frac{2N}{3}\right] = Pr\left[|Y| > \frac{2N + 1}{3}\right] = Pr\left[|Y| > \frac{\sqrt{2}}{3} \times \frac{2N + 1}{\sqrt{2}}\right] \leq \frac{2}{(2N + 1)^2}
			\end{equation}
			\item .
		\end{enumerate}

	\section{Hashing}
		\begin{enumerate}
			\item .
			\item .
			\item .
			\item .
		\end{enumerate}

	\section{Election prediction}
		\begin{enumerate}
			\item .
			\item .
		\end{enumerate}

	\section{Estimating mean and median}
		\begin{enumerate}
			\item .
			\item .
		\end{enumerate}

	\section{Randomized Min-Cut}
		\begin{enumerate}
			\item .
			\item .
			\item .
			\item .
		\end{enumerate}
\end{document}
