\documentclass{article}

\usepackage{graphicx}
\usepackage{float}
\usepackage{amsthm}
\usepackage{algorithm}
\usepackage{algorithmicx}
\usepackage{algpseudocode}
\usepackage{amsmath}

\usepackage{amssymb}
\newenvironment{claim}[1]{\par\noindent\underline{Claim:}\space#1}{}
\newenvironment{guess}[1]{\par\noindent\underline{Guess:}\space#1}{}
\usepackage{amsthm}

\usepackage{geometry}
\geometry{left=2.5cm,right=2.5cm}

\title{Homework5}
\author{Qinyun Song}
\date{}

\begin{document}
	\maketitle

	\section{Markov's inequality}
		For a random variable $X$ and $X = -9$ with probability $0.5$ while $X = 11$ with probability $0.5$. Then we can see that \begin{equation}
		E[X] = 0.5 \times -9 + 0.5 \times 11 = 1
		\end{equation}
		And the probability \begin{equation}
			Pr \left[ X >  10 \right] = 0.5 = \frac{1}{2}
			\end{equation}

	\section{Comparing concentration inequalities}
		\begin{enumerate}
			\item The expectation of $X$ is \begin{equation}
				E\left[X\right] = 1 \times \frac{1}{3} + 0 \times \frac{2}{3} = \frac{1}{3}
				\end{equation}
				Then from the \emph{Markov's inequality}, we can see that \begin{equation}
				Pr \left[ X > \frac{t}{3}\right] \leq \frac{1}{t}
				\end{equation}
			\item We can calculate the variance of $X$ as \begin{equation}
					var(X) = N \times \frac{1}{3} \times \frac{2}{3} = \frac{2N}{9}
				\end{equation}
			\item Define variable $y_i = x_i - \frac{1}{3}$. Then we can see that $E\left[y_i\right] = 0$. We can see that \begin{equation}
			y_i = \frac{2}{3} wp \frac{1}{3} and y_i = -\frac{1}{3} wp \frac{2}{3}
			\end{equation}
			Thus we know that \begin{equation}
				var(Y) = E\left[Y^2\right] = \frac{4}{9} \times \frac{1}{3} + \frac{1}{9} \times \frac{2}{3} = \frac{6}{27} = \frac{2}{9}
			\end{equation}
			So for the std-dev, we have \begin{equation}
				std-dev(Y) = \sqrt{var(Y)} = \frac{\sqrt{2}}{3}
			\end{equation}
			If we want $X > \frac{2N}{3}$, then we need $Y > \frac{2N + 1}{3}$. Then by using the \emph{Chebychev's inequality}, we know that \begin{equation}
				Pr\left[ X > \frac{2N}{3}\right] = Pr\left[|Y| > \frac{2N + 1}{3}\right] = Pr\left[|Y| > \frac{\sqrt{2}}{3} \times \frac{2N + 1}{\sqrt{2}}\right] \leq \frac{2}{(2N + 1)^2}
			\end{equation}
			\item .
		\end{enumerate}

	\section{Hashing}
		\begin{enumerate}
			\item .
			\item .
			\item .
			\item .
		\end{enumerate}

	\section{Election prediction}
		\begin{enumerate}
			\item Define variable $x_i$ as the $i-th$ person vote $1$ or $0$. It satisfies the following function \begin{equation}
				x_i = 1 if vote for 1 = -1 if vote for 0
				\end{equation}Then we know that \begin{equation}
				E\left[X\right] = 1 \times p  -1 \times (1 - p) = 2p - 1
			\end{equation}
				Define another variable $x_i' = x_i - (2p - 1)$. Then we know that $E[X'] = 0$. We can also know that \begin{equation}
					E\left[X'\right] = 4 \times (p(1-p)^2 + (1 - p)p^2) = 4p(1 - p)
				\end{equation}
				So by using the \emph{Chernoff inequality}, we know that \begin{equation}
					Pr\left[\sum_ix_i' >t \right] \leq 2e^{\frac{-t^2/2}{4np(1 - p) + t}}
				\end{equation}
				Since \begin{equation}
					\sum_i x_i' = \sum_i x_i - n(2p - 1)
				\end{equation}
				we know that \begin{equation}
					Pr\left[\sum_i x_i' > t\right] = Pr\left[\sum_i x_i > t + n(2p - 1)\right]
				\end{equation}
				Since we want to know if $\sum_i x_i = 0$ or not, we then need \begin{equation}
					t = -n(2p - 1)
				\end{equation}
				So we know that \begin{equation}
					Pr\left[\sum_ix_i > 0\right] \leq 2e^{\frac{-(-n(2p - 1))^2 / 2}{4np(1 - p) - n(2p - 1)}}
					= 2e^{\frac{-n(2p - 1)^2}{4p - 4p^2 + 1}}
				\end{equation}
				Since $p = 0.75$, we know that \begin{equation}
					Pr\left[\sum_ix_i > 0\right]\leq 2e^{\frac{-0.25n}{1.75}} = 2e^{-\frac{n}{7}}
				\end{equation}
				If we ned at least $99\%$ confidence, we will require the right part of the equation be no less than $99\%$. So we want that \begin{equation}
					2e^{-\frac{n}{7}} \geq 99\%
				\end{equation} Finally we can see that, to ensure the confidene no less than $99\%$, we need \begin{equation}
					n \geq 5
				\end{equation}
				So the bound of $n$ is no less than $5$.
			\item If the probability now is $p = 0.501$, then we know that \begin{equation}
				Pr\left[ \sum_i x_i > 0 \right] \leq 2e^{\frac{-0.000004n}{1.999996}} = 2e^{-\frac{n}{499999}}
			\end{equation}
			Then we need \begin{equation}
				2e^{-\frac{n}{499999}} \geq 99\%
			\end{equation}
			So we need that \begin{equation}
				n \geq 351599
			\end{equation}
		\end{enumerate}

	\section{Estimating mean and median}
		\begin{enumerate}
			\item For the case $n = 3$ and $k = 2$, define the empirical average as \begin{equation}
				u'_{i_1,i_2} = \frac{1}{2}(A\left[i_1\right] + A\left[i_2\right]) 
			\end{equation}
			So we have \begin{align}
				E\left[u'\right] &= \frac{1}{3} u'_{1, 2} + \frac{1}{3} u'_{2, 3} + \frac{1}{3}u'_{1, 3} \\
					&= \frac{1}{3}  \sum_{i = 1}^3 A\left[i\right]\\
					&= u
			\end{align}
			%If $k = 1$, the variance is calculated as \begin{align}
			%	var & = E\left[(u' - u)^2\right] \\
			%			& = \frac{1}{n}\sum_{i = 1}^n{ (A\left[i\right] - u)^2} \\
			%				& = \frac{1}{n}\sum_{i = 1}^n {A\left[i\right]^2 - u^2} \\
			%					& \leq \frac{1}{n} - u^2 
			%\end{align}
			Then if $k = 2$, the variance can now be calculated as \begin{align}
				var &= E\left[(u' - u)^2 \right] \\
					   &= \frac{1}{3}(u'_{1, 2} - u)^2 + \frac{1}{3}(u'_{2, 3} - u)^2 + \frac{1}{3}(u'_{1, 3} - u)^2 \\
					   &= \frac{1}{6}(\sum_{i = 1}^3 A\left[i\right]^2 - A\left[1\right]A\left[2\right] - A\left[2\right]A\left[3\right] - A\left[1\right]A\left[3\right])
				\end{align}
			Compare with the replacement case, in this case, we disregard the case that we choose one indice twice. 
			\item .
		\end{enumerate}

	\section{Randomized Min-Cut}
		\begin{enumerate}
			\item .
			\item .
			\item .
			\item .
		\end{enumerate}
\end{document}
